%
% Szakdolgozatminta az Eszterházy Károly Katolikus Egyetem
% matematika illetve informatika szakos hallgatóinak.
%

\documentclass[
% opciók nélkül: egyoldalas nyomtatás, elektronikus verzió
% twoside,     % kétoldalas nyomtatás
% tocnopagenum,% oldalszámozás a tartalomjegyzék után kezdődik
]{thesis-ekf}
\usepackage[T1]{fontenc}
\usepackage{hulipsum}
\PassOptionsToPackage{defaults=hu-min}{magyar.ldf}
\usepackage[magyar]{babel}
\usepackage{mathtools,amssymb,amsthm,pdfpages}
\footnotestyle{rule=fourth}

\newtheorem{tetel}{Tétel}[chapter]
\theoremstyle{definition}
\newtheorem{definicio}[tetel]{Definíció}
\theoremstyle{remark}
\newtheorem{megjegyzes}[tetel]{Megjegyzés}

\begin{document}
\institute{Matematikai és Informatikai Intézet}
\title{Programozható elektronikák alkalmazásai tervdokumentáció}
\author{Bagoly Gábor\\programtervező informatikus}
\supervisor{Dr. Geda Gábor\\egyetemi docens}
\city{Eger}
\date{2022}
\maketitle
\tableofcontents

\chapter*{Bevezetés}
\addcontentsline{toc}{chapter}{Bevezetés}
\section*{Kis összegzés jelenleg}

Témám egy okos otthon tervezése és elkészítése, ahol különböző okos eszközök irányításával egy lokális szerveren
úgy, hogy azt egy Raspberry Pi, vagy NodeMCU szolgáltatna, és irányítaná az eszközöket.
Az eszközökkel például világítást, fűtést is lehetne vezérelni, akár automatikusan. Az eszközök
különböző féle programozható elektronikák segítségével lesznek megvalósítva.
Ezen eszközök alkatrészei közé fognak tartozni különféle szenzorok is, amik segítségével
megkönnyíthető az automatizálás.
\section*{Cél}
Célom az lenne ezzel, hogy belelássak az okos otthonok működésébe, különböző programozható
eszközök alkalmazását jobban megismerjem, és, hogy egy olyan általános kezelőfelületet
tudjak létrehozni, amit könnyen tud a felhasználó alkalmazni.

\chapter*{Terv}
\begin{enumerate}
	\item Bevezetés bővebb megírása.
	\item Okos otthonról leírás röviden.
	\item Raspberry története, felépítése, különböző fajták. Alkalmazott Raspberry típusa, felépítése.
	\item Arduino eszközök története, rövid leírás, alkalmazási módok. Alkalmazott arduino eszközök használata.
	\item Az elkészített okos otthonról egy ábra, (először csak terv, ez alapján el lehet indulni), alkalmazott adatbázis ábra. (PlantUML-ben az adatbázis, az okos otthon terve DrawIO, PlantUML vagy saját készítésű rajz)
	\item Eszközök és szoftver alkalmazása:
		\begin{enumerate}
			\item Relé / FET használata.
			\item MQTT használata/működése.
			\item NodeRED használata.
			\item Laraveles kezelőfelület leírása.
			\item Ezeknek az együttes kommunikációja.
		\end{enumerate}
	\item Az elkészített eszközöknek az együttes alkalmazása.
	\item A kis "babaház" / házmodellben való alkalmazása az eszközöknek.
	\item Elkészíteni kívánt rendszer:
	\begin{enumerate}
		\item Lámpa alkalmazása.
		\item Egy (vagy több) eddig nem "okos" eszköz "okosítása".
		\item Hőmérséklet és páratartalom alapján automatikus hőszabályozás.
		\item Statiszkikai oldal, ahol a szenzoros adatokat megjelenítem.
		\item Megfelelő adatok adatbázisba való elmentés. - hőmérséklet/páratartalom.
		\item Mindezt egy Laravel keretrendszerben megírt webalkalmazáson keresztül. - ezt azért választottam úgy, mert így ezt bármely operációs rendszeren keresztül (legyen az iOS, android, Windows stb...) el lehet érni, csak webböngészővel kell rendelkezni.
		\item A szervert egy Raspberry Pi 4B segítségével valósítom majd meg.
		\item CyPress használatával automatizált webfelület tesztelés.	
	\end{enumerate}
\end{enumerate}


\section*{Az Arduino platform}
Az Arduino egy széleskörűen elterjedt, nyílt forrású fejlesztőplatform. Az Arduino-t oktatási céllal hozták létre, de előszeretettel használják otthoni projektekhez, a kisebb automatizálási feladatoktól kezdve az okos otthonok kialakításáig. Ezek mellett ma már az ipari alkalmazások sem ritkák és rengeteg IoT
Az Arduino platform része egy elektronikai áramköri lap és a szoftveres környezet. Maga az áramköri lap nagyon sokféle felépítésben megtalálható.

\emph{ide később folytatni azokkal, amit fentebb is leírtam}







\chapter*{Összegzés}
\addcontentsline{toc}{chapter}{Összegzés}
Tapasztalatok leírása itt, milyen tervek vannak ezek után a jövőben ezzel a projekttel. Milyen módon lehet ezen még javítani, fejleszteni.
\end{document}